\documentclass{article}
\usepackage[T1]{fontenc}
\usepackage[utf8]{inputenc}
\usepackage[margin=1in]{geometry}
\usepackage{fancyhdr} 
\usepackage{listings}
\usepackage[ruled,vlined]{algorithm2e}
\usepackage{amsthm}
\usepackage{amsfonts}
\usepackage{amssymb}
\usepackage{graphicx}
\usepackage[dvipsnames]{xcolor}
\usepackage{xy}
% \usepackage{url} % Commented out because hyperref provides similar functionality
\usepackage{parskip}
\usepackage{comment}
\usepackage{setspace}
\usepackage{enumerate}
\usepackage{multirow}
\usepackage{hyperref}
\usepackage{caption}
\usepackage{subcaption}
\usepackage{booktabs}
\usepackage{wrapfig}
\usepackage{times}

\captionsetup[figure]{font={small,it}}

\usepackage[backend=biber,style=numeric,sortcites,maxbibnames=99]{biblatex}
\addbibresource{references.bib}

\newcommand{\HRule}{\rule{\linewidth}{0.5mm}}
\newcommand{\Hrule}{\rule{\linewidth}{0.3mm}}
\newcommand{\classnum}{CS-GY 6313 B}

\makeatletter% since there's an at-sign (@) in the command name
\renewcommand{\@maketitle}{%
  \parindent=0pt% don't indent paragraphs in the title block
  \centering
  {\Large \bfseries\textsc{\@title}}
  \HRule\par%
  \textit{\@author \hfill \classnum}
  \par
}
\makeatother% resets the meaning of the at-sign (@)

\title{Assignment 3: Interactive Visualization} 
\author{Ivan Aristy — iae225}
% \classnum

\begin{document}
  \maketitle % prints the title block
  \thispagestyle{empty}
  % \vspace{-15pt}

\section{Interactive Visualization}
\label{sec:sec1}

\subsection{Question }
\label{subsec:subsec1}

\textbf{How has my champion's win rate changed over time? Is my champion still strong in the current meta?}

I want the user to easily see the win rate of their champion over time. 
Winrate is a key metric in League of Legends, as it is a good indicator of how strong a champion is in the current meta.
This would allow the user to see how their champion has been impacted in recent patches.

Winrate is not a direct representation of power, since some champions are harder to play than others.
Easy champions usually have high winrates, while hard champions have lower winrates.
Additionally, hyper-specific champions have really high winrates, since they are only played in specific situations.

Nevertheless, patterns in winrates can be indicative of power. For example, a 2\% growth in winrates
over a patch can be indicative of a very strong relating buff, while a 2\% decrease can be indicative of a strong relative nerf.
Additionally, a winrate that is consistently above 50\% can show that a champion has and is strong in the current meta.

\subsection{Data}
\label{subsec:subsec2}

\subsubsection{Data Source}

Riot's Developer API is very hard to use. In my opinion, is not well documented, and it is very hard to get the data you want.
Hence, I wll scrape some data off the internet to get the winrate of a champion over time.

Particularly I need:
\begin{enumerate}
  \item The winrate of a champions over time.
  \item The tier of the champion over time.
  \item Current highest winrate build of the champion.
  \item Visual Assets of the champion.
  \item Visual Assets of items.
  \item Patch history.
\end{enumerate}



% \begin{figure}[ht] % Change the position of your figure https://www.overleaf.com/learn/latex/Positioning_images_and_tables
%   \centering
%   \includegraphics[width=0.75\textwidth]{figs/Trump Chart.png}
%   \caption{
%       Trump's Illegal Immigration Chart
%   }
%   \label{fig:fig1}
% \end{figure}




\begin{refcontext}[sorting=nyt]
\printbibliography
\end{refcontext}

\end{document}

