\documentclass{article}
\usepackage[T1]{fontenc}
\usepackage[utf8]{inputenc}
\usepackage[margin=1in]{geometry}
\usepackage{fancyhdr} 
\usepackage{listings}
\usepackage[ruled,vlined]{algorithm2e}
\usepackage{amsthm}
\usepackage{amsfonts}
\usepackage{amssymb}
\usepackage{graphicx}
\usepackage[dvipsnames]{xcolor}
% \usepackage{xy}
% \usepackage{url} % Commented out because hyperref provides similar functionality
\usepackage{parskip}
\usepackage{comment}
\usepackage{setspace}
\usepackage{enumerate}
\usepackage{multirow}
\usepackage{hyperref}
\usepackage{caption}
\usepackage{subcaption}
\usepackage{booktabs}
\usepackage{wrapfig}
\usepackage{times}

\captionsetup[figure]{font={small,it}}

\usepackage[backend=biber,style=numeric,sortcites,maxbibnames=99]{biblatex}
\addbibresource{references.bib}

\newcommand{\HRule}{\rule{\linewidth}{0.5mm}}
\newcommand{\Hrule}{\rule{\linewidth}{0.3mm}}
\newcommand{\classnum}{CS-GY 6313 B}

\makeatletter% since there's an at-sign (@) in the command name
\renewcommand{\@maketitle}{%
  \parindent=0pt% don't indent paragraphs in the title block
  \centering
  {\Large \bfseries\textsc{\@title}}
  \HRule\par%
  \textit{\@author \hfill \classnum}
  \par
}
\makeatother% resets the meaning of the at-sign (@)

\title{Assignment 2: Your title here}
\author{Your name here}
% \classnum

\begin{document}
  \maketitle % prints the title block
  \thispagestyle{empty}
  % \vspace{-15pt}

References to papers and books look like this \cite{munzner2014visualization}. 

References to sections look like this: \autoref{sec:sec1}

References to figures look like this: \autoref{fig:fig1}

\begin{figure}[ht] % Change the position of your figure https://www.overleaf.com/learn/latex/Positioning_images_and_tables
    \centering
    \includegraphics[width=0.75\textwidth]{figs/sight.jpg}
    \caption{
        Figure captions look like this. You can change the size of the figure using the \texttt{width} parameter in the \texttt{\textbackslash includegraphics} command. You can change the position of the figure using the position arguments in the \texttt{\textbackslash begin\{figure\}} environment command.
        URLs look like this: \url{https://www.mfa.org/exhibition/michaelina-wautier-and-the-five-senses}
    }
    \label{fig:fig1}
\end{figure}

\section{Section 1}
\label{sec:sec1}
Example section.

\begin{refcontext}[sorting=nyt]
\printbibliography
\end{refcontext}

\end{document}

