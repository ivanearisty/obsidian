\documentclass{article}

\usepackage[utf8]{inputenc}
\usepackage{amsmath}
\usepackage{graphicx}
\usepackage{hyperref}
\usepackage{parskip}

\title{Assignment 1}
\author{Ivan Aristy — iae225}
\date{\today}

\begin{document}

\maketitle

\section{Assignment}
\textbf{Assignment 1: Designing a Static Visualization}

Your task for this assignment is to design a static visualization that you believe effectively communicates an idea or message about the sunshine data, and provide a written report (no more than 1 page, single spaced, not including images) that details your design. Note that you should include at least one image of your visualization in your report (i.e., when all of the images are excluded from the report, the text does not exceed 1 page). We recommend that you start this assignment by identifying a question about the data that you would like your visualization to answer. Then, you should design a visualization to answer that question, and use that question as the title of your visualization.

You are required to use the given data set, but you are free to transform the data in any way that you wish. That is, you can manipulate the data using transformations such as a log transformation, computing percentages or averages, grouping elements into categories, or removing unnecessary variables. You are also allowed to incorporate additional data from external sources, but if you do so, you need to make a note of it in your submission.

Remember that different visualizations can emphasize different aspects of a data set, so your writeup should include details about which components of the data you intended to communicate. You should also provide comments on which aspects of the data set are obscured by your visualization design.

Your report should explain the rationale behind your design decisions. You should document the visual encodings you used and why they are effective for communicating your intended message. These decisions include things like the choice of visualization type, size, color, scale, and other visual elements (which we will learn more about during Week \#3 and Week \#4 lectures).

\newpage

\subsection{Intended Visualization \& Question}

\textbf{What are the best cities for nice weather and lots of sunshine?}

This is the question we aim to answer in this paper. In a more nuanced sense,
the question can be interpreted as "What city is the 'winner' and the runner-up between the 72 and 84°F
range, while maximizing for sunshine hours?" The question assumes only temperatures during daytime, since this 
is when most people are awake and active.

The reason for this question is that the 72-84°F range is considered the most comfortable for most people. 


\subsection{Preliminary Analysis}
\subsubsection{Considerations}
\subsubsection{Data Cleaning \& Gathering}

\subsection{Iterations}

\subsection{Final Visualization}
\subsubsection{Does this answer the question?}
\subsubsection{Encoding (Marks/Channels, Redundancy, Color/Accessibilty)}
\subsubsection{Highlighted vs Obscured Data}

\bibliographystyle{plain}
\bibliography{references}

\end{document}